\documentclass[11pt,twoside]{article}
\usepackage{asp2014}

\aspSuppressVolSlug
\resetcounters

\bibliographystyle{asp2014}

\markboth{Mueller et al.}{Qserv Distributed Petascale Database}

\begin{document}
%\ssindex{LSST}
%\ssindex{observatories!ground-based!LSST}
%\ssindex{observatories!ground-based!Rubin}

\title{Qserv: A Distributed Petascale Database for the LSST Catalogs}

%% DO NOT EDIT THIS FILE. IT IS GENERATED FROM db2authors.py"
%% Regenerate using:
%%    python $LSST_TEXMF_DIR/bin/db2authors.py > authors.tex


\author{Fritz~Mueller$^1$ }
\affil{$^1$SLAC National Accelerator Laboratory,  2575 Sand Hill Rd., Menlo Park, CA 94025, USA}
\paperauthor{Fritz~Mueller}{}{None}{SLAC National Accelerator Laboratory}{}{ Menlo Park}{CA}{94025}{ USA}
% Yes they said to have these index commands commented out.
%\aindex{Mueller,Fritz}


\begin{abstract}
Qserv is a distributed, shared-nothing, SQL database system being developed by the Vera Rubin Observatory to
host the multi-petabyte astronomical catalogs that will be produced by the LSST survey.  Here we sketch the
basic design and operating principles of Qserv, and provide some updates on recent developments.
\end{abstract}

\section{Introduction}

The Legacy Survey of Space and Time \citep{2019ApJ...873..111I} is a "deep fast wide" optical/near-IR survey
of half the sky in \emph{ugrizy} bands to \emph{r} 27.5 (36\,nJy) based on 825 visits over a 10-year period,
to be carried out by the Vera C.\ Rubin Observatory in Chile.

The astronomical catalogs to be produced by the survey are notionally described in \citet{LSE-163}, and the
corresponding database schema is described in \citet{LDM-153}.  By the 10th year of the survey, the catalog
database is expected to run to approximately 60 trillion rows, requiring more than 10 petabytes of storage
without consideration of replication or indices.

\acknowledgements This material or work is supported in part by the National Science Foundation through
Cooperative Agreement AST-1258333 and Cooperative Support Agreement AST1836783 managed by the Association of
Universities for Research in Astronomy (AURA), and the Department of Energy under Contract No.
DE-AC02-76SF00515 with the SLAC National Accelerator Laboratory managed by Stanford University.

\bibliography{C15}

\end{document}
